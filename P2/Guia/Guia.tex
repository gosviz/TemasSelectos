\documentclass[letterpaper]{article}
\usepackage[spanish,mexico]{babel}
\usepackage[T1]{fontenc}
\usepackage[utf8]{inputenc}
\usepackage[pdftex]{graphicx}
\usepackage{listings}
\lstset{language=C, basicstyle=\footnotesize,captionpos=b, breaklines=true,
tabsize=4}
\lstset{literate=
  {á}{{\'a}}1 {é}{{\'e}}1 {í}{{\'i}}1 {ó}{{\'o}}1 {ú}{{\'u}}1
  {Á}{{\'A}}1 {É}{{\'E}}1 {Í}{{\'I}}1 {Ó}{{\'O}}1 {Ú}{{\'U}}1
  {ä}{{\"a}}1 {ë}{{\"e}}1 {ï}{{\"i}}1 {ö}{{\"o}}1 {ü}{{\"u}}1
  {Ä}{{\"A}}1 {Ë}{{\"E}}1 {Ï}{{\"I}}1 {Ö}{{\"O}}1 {Ü}{{\"U}}1
}


\title{Universidad Nacional Autónoma de México\\
Facultad de Ingeniería\\}
      
\author{\\
  	\\
	\\
	\\
	Sánchez Monterde Luis Antonio\\
	Vizcayno García José Agustín
  	\\
	Grupo:4\\
	Laboratorio: Bio-Robótica\\
	Temas selectos de control y robótica\\
      \\
    \\
  \\
\\
	Práctica 2.- Conexión de Sensores a la Tarjeta\\
      de Arduino, Parte 1\\
    \\
  \\
  \\
  \\
}
\date{Miercoles 12 de septiembre de 2018}

\begin{document}
\newcommand{\figura}[3]{
\begin{figure}[hbtp]
  \centering
  \includegraphics[width=#3\textwidth]{#1}
  \caption{#2}
  \label{fig:#1}
\end{figure}
}
\pagenumbering{gobble} 
\maketitle
\clearpage
\section{Objetivo}
\begin{itemize}
  \item Familiarizar al alumno con la conexión de diferentes sensores a 
    la tarjeta  Arduino.
\end{itemize}
\pagenumbering{arabic}
\section{Desarrollo}
Se requiere crear un programa que lea cinco sensores, dos digitales y tres
analógicos, se hacen las conexiones como indica el formato de la práctica y
se les asigna a los pines resumidos en la tabla \ref{tab:pines}.
\begin{table}
  \centering
  \begin{tabular}{c|c|l}
    Sensor & Pin & Descripción \\
    \hline
    \hline
    contact & 2 & Push button\\
    photord & 3 & Foto resistencia digital\\
    temp & A0 &  Sensor de temperatura\\
    photora & A1 & Foto resistencia analógica\\
    infrared & A2 & Sensor infrarrojo
  \end{tabular}
  \caption{Tabla de asignación}
  \label{tab:pines}
\end{table}

Para leer dichos sensores se realizo un programa en C, el programa se 
encuentra en el listado \ref{lst:primer}
\begin{lstlisting}[caption=Programa para sensar, label=lst:primer]
#include<string.h> //para strcmp() compara cadenas

//Inicialización y declaración de variables
short int contacto = 0,
          foto_Resistencia_Digital = 0;
double  temperatura = 0,
        foto_Resistencia_Analogica = 0,
        infrarojo = 0;
char buff[20];	//Buffer serial;

void setup(){
  //Inicialización de puertos y pines
  pinMode(2,INPUT);
  pinMode(3,INPUT);
  pinMode(A0,INPUT);
  pinMode(A1,INPUT);
  pinMode(A2,INPUT);
  
  Serial.begin(9600);   //Habilita comunicación serial
  Serial.setTimeout(100); //100ms máximo para caracter nulo
}

void loop(){

}

void serialEvent(){ //Espera a que existan cambios en el puerto
  if(Serial.available()){ //Sí, hay datos en el buffer
    //Carga los datos en un buffer auxiliar
    Serial.readBytesUntil('\n', buff, 20);
    //Serial.println(buff);
    if(!strcmp(buff, "shs contact")){//Compara el buffer 
      contacto = digitalRead(2); //guarda el valor del pushbotton
      Serial.print("contacto ");
      Serial.println(contacto);
    }else if(!strcmp(buff, "shs temp")){
      temperatura = analogRead(A0);
      Serial.print("Temperatura ");
      Serial.println(temperatura);
    }else if(!strcmp(buff, "shs photord")){
      foto_Resistencia_Digital = digitalRead(3);
      Serial.print("Fotoresistencia Digital ");
      Serial.println(foto_Resistencia_Digital);
    }else if(!strcmp(buff, "shs photora")){
      foto_Resistencia_Analogica = analogRead(A1);
      Serial.print("Foto Resistencia Analógica ");
      Serial.println(foto_Resistencia_Analogica);
    }else if(!strcmp(buff, "shs infrared")){
      infrarojo = analogRead(A2);
      Serial.print("Infrarojo ");
      Serial.println(infrarojo);
    }else{
        Serial.println("Ningun sensor con esa descripción");
    }
    memset(buff,0,sizeof(buff));//Limpia el buffer
  }
}
\end{lstlisting}
Los detalles importantes de este programa son pocos, el primero es que
se opto por un manejo por eventos en lugar de uno en modo poleo, el
segundo es que se utiliza un buffer auxiliar para leer la cadena de 
entrada. El programa funciona a la perfección por lo que solo queda
agregar una rutina que pueda calcular la distancia más próxima a la 
verdadera de acuerdo a nuestro sensor.

Para encontrar el valor más cercano a la distancia verdadera, se utiliza
el método de regresión lineal por mínimos cuadrados, en nuestro caso
con R, para ello se toman muestras espaciadas, en nuestro caso 
muestras cada $5 cm$ como en el listado \ref{lst:muestras}
\begin{lstlisting}[caption=Muestra de las muestras, label=lst:muestras]
30      34.58
30      34.58
30      34.58
30      34.58
30      35.03
30      34.58
30      35.03
30      34.13
30      34.58
30      34.58
30      34.13
30      34.58
30      34.58
30      34.58
30      34.58
30      35.03
30      35.03
30      34.58
.	.
.	.
.	.
\end{lstlisting}
las mediciones reales correspondientes a cada $5cm$ en función del ADC 
se muestran en la tabla \ref{tab:distReales}
\begin{table}
  \centering
  \begin{tabular}{c|c}
    Distancia [cm] & Valor real del sensor \\
    \hline
    \hline
    5 & 483 \\
    10 & 842 \\
    15 & 939 \\
    20 & 975 \\
    25 & 991 \\
    30 & 998 \\
  \end{tabular}
  \caption{Valores reales}
  \label{tab:distReales}
\end{table}
Con esta información se realizo la regresión lineal correspondiente para
polinomios de primero a cuarto orden, con la ayuda del script en R 
proporcionado en clase,
\begin{lstlisting}[language=R]
#Se crea un función que actúa como el polinomio y se le nombra m3
m3=lm(formula = x ~ I(y)+I(y^2)+I(y^3))

#se calcula los coeficientes, con funciones propias de R
a0=m3$coef[1]
a1=m3$coef[2]
a2=m3$coef[3]
a3=m3$coef[4]

#Asignación simple de una función, con los coeficientes correspondientes
yc3 = a0 + a1*y + a2*y^2 + a3*y^3

#Imprime los coeficientes
cat("\ntercero a0= ",a0, " a1=", a1, " a2=", a2, "a3= ", a3)
\end{lstlisting}

Se tienen los coeficientes requeridos, para ver su comportamiento se
grafican en las figuras \ref{fig:primer} a \ref{fig:cuarto}, 
\figura{primer}{Polinomio de primer orden}{0.7}
\figura{segundo}{Polinomio de segundo orden}{0.7}
\figura{tercer}{Polinomio de tercer orden}{0.7}
\figura{cuarto}{Polinomio de cuarto orden}{0.7}

Los coeficientes para cada polinomio son:
\begin{itemize}
  \item Primer orden
    \begin{eqnarray*}
      a_0 &= 0.3117388\\
      a_1 &= 0.8640718
    \end{eqnarray*}
  \item Segundo orden
    \begin{eqnarray*}
      a_0 &= -0.7533509\\
      a_1 &= 1.001657\\
      a_2 &= -0.003390901
    \end{eqnarray*}
  \item Tercer orden
    \begin{eqnarray*}
      a_0 &= -1.494873\\
      a_1 &= 1.160523\\
      a_2 &= -0.01246657\\
      a_3 &= 0.0001494483 
    \end{eqnarray*}
  \item Cuarto orden
    \begin{eqnarray*}
      a_0 &= 0.0505364\\
      a_1 &= 0.7056092\\
      a_2 &= 0.029781\\
      a_3 &= -0.00138495\\
      a_4 &= 1.898157e-05
    \end{eqnarray*}
\end{itemize}
por simple inspección de las figuras se ve que las dos primeras son 
prácticamente iguales, es obvio debido a que los coeficientes son muy 
parecidos, por otro lado los polinomios de tercer y cuarto orden se ven 
más ``curvos'', haciendo pruebas con cada uno de los polinomios,
determinamos que el de cuarto orden es el más adecuado. 

Se agrega al programa, reemplazando el código
\begin{lstlisting}
infrarojo = analogRead(A2);
Serial.print(``Infrarojo '');
Serial.println(infrarojo);
\end{lstlisting}
por

\begin{lstlisting}
//a0, a1, a2, a3, a4, son los coeficientes corregidos
x = 13*pow((analogRead(A2)*0.0048828115),-1);
y = a0+a1*x+a2*pow(x,2)+a3*pow(x,3)+a4*pow(x,4);
Serial.print("Infrarojo ");
Serial.println(y);
\end{lstlisting}

\end{document}
